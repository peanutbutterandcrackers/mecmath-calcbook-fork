\addchap{Preface}
This book covers calculus of a single variable. It is suitable for a year-long
(or two-semester) course, normally known as Calculus I and II in the United
States. The prerequisites are high school or college algebra, geometry and
trigonometry. The book is designed for students in engineering, physics,
mathematics, chemistry and other sciences.

One reason for writing this text was because I had already written its sequel,
\emph{Vector Calculus}. More importantly, I was dissatisfied with the current
crop of calculus
textbooks, which I feel are bloated and keep moving further away from the
subject's roots in physics. In addition, many of the intuitive
approaches and techniques from the early days of calculus---which I think often
yield more insights for students---seem to have been lost.

I agree with the views of the late
Russian mathematician V.I. Arnold on teaching mathematics, in particular the
idea that ``Mathematics is the part of physics where experiments are
cheap.''\footnote{\textsc{Arnold, V.I.},  ``On Teaching Mathematics'',
\emph{Russian Math. Surveys} 53 (1998), No. 1, 229-236. An HTML version is at
\url{https://www.uni-muenster.de/Physik.TP/~munsteg/arnold.html}}
The ties to physics are especially important in calculus, so this book tries to
introduce new concepts with physical motivations (what other motivations
can there be?). The book contains exercises and
examples that I hope will adequately prepare students who continue on in physics
and engineering.\footnote{The book covers some of the types of problems and
techniques for solving them that such students will likely encounter. Facility
with using named constants (e.g. $c$, $h$, $T$) is also emphasized.}

Perhaps controversially, the book uses infinitesimals, making it a
bit of a ``throwback'' or ``retro'' calculus text. My justification for this
heretical act was purely pedagogical: infinitesimals make learning calculus
easier, and their use aligns more with the way students will see calculus
in their physics, chemistry and other science classes and textbooks (where
infinitesimals are employed liberally). This might ruffle some feathers among
mathematical ``purists,'' but they are not the main audience for this book. That
said, the book is still compatible with the usual limit-based approach, so
an instructor could simply ignore the parts involving infinitesimals and teach
the material as he or she normally would. I did not want to be dogmatic, so I
used infinitesimals where I thought it made sense, and used limits where
appropriate (e.g. in discussing continuity, series). Again, pedagogy was my
priority.

The exercises at the end of each section are divided into three categories: A, B
and C. The A exercises are mostly of a routine computational nature, the B
exercises are slightly more involved, and the C exercises usually require some
effort or insight to solve. A crude way of describing A, B and C would be
``Easy'', ``Moderate'' and ``Challenging'', respectively. However, many of the B
exercises are easy and not all the C exercises are difficult.  Appendix A
provides answers and hints to many of the odd-numbered and some of the
even-numbered exercises.

A few exercises require the student to write a computer program to solve
numerical approximation problems (e.g. numerical methods for approximating
definite integrals). Algorithms are presented in pseudocode, with code
implementations in various languages (primarily Java, but also Python,
Octave, Sage). I hope the code comments will help the reader figure out what is
being done, regardless of familiarity with those languages. Students are free to
implement solutions using the language of their choice. There are no dedicated
``calculator exercises,'' as those have been rendered pointless by modern
computing (with which students need to become acquainted).

Stylistically I made a conscious effort to break from an unfortunate but all too
common mode of writing in mathematics texts, lamented in the preface of a
physics book: ``Nothing is more repellent to normal human beings than the
clinical succession of definitions, axioms, and theorems generated by the
labours of pure mathematicians.''\footnote{\textsc{Ziman, J.M.},
\emph{Elements of Advanced Quantum Theory}, Cambridge, U.K.: Cambridge
University Press, 1969.} I have been guilty of that sin myself, but I have
changed my ways and banished all traces of that sort of thing from this book. So
you won't find Definition 1.2, Theorem 3.3, Corollary 4.6, Lemma 5.7, Axiom 1B,
etc. Instead, I tried to borrow the best of the styles from the physics and
foreign languages textbooks I enjoyed so much in college. I also deliberately
avoided what the author Gore Vidal called the ``we-ness'' that
prevails in academic writing. There is no good reason for the ``royal we'' in a
textbook, and it comes off as a bit pompous, so \emph{we} won't use it.

This book is released under the GNU Free Documentation License (GFDL), which
allows others to not only copy and distribute the book but also to modify it.
For more details, see the included copy of the GFDL. So that there is no
ambiguity on this matter, anyone can make as many copies of this book as desired
and distribute it as desired, without needing my permission. The PDF version
will always be freely available to the public at no cost (go
to \url{http://www.mecmath.net/calculus}). Feel free to contact me at
\texttt{\href{mailto:mcorral@schoolcraft.edu}{mcorral@schoolcraft.edu}} for any
questions on this or any other matter regarding the book. I welcome your
feedback.

\begin{flushleft}
\emph{Schoolcraft College}\hspace{\stretch{1}}\textsc{Michael Corral}\\
\emph{December 2020}
\end{flushleft}
